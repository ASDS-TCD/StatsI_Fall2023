\documentclass[12pt,letterpaper]{article}
\usepackage{graphicx,textcomp}
\usepackage{natbib}
\usepackage{setspace}
\usepackage{fullpage}
\usepackage{color}
\usepackage[reqno]{amsmath}
\usepackage{amsthm}
\usepackage{fancyvrb}
\usepackage{amssymb,enumerate}
\usepackage[all]{xy}
\usepackage{endnotes}
\usepackage{lscape}
\newtheorem{com}{Comment}
\usepackage{float}
\usepackage{hyperref}
\newtheorem{lem} {Lemma}
\newtheorem{prop}{Proposition}
\newtheorem{thm}{Theorem}
\newtheorem{defn}{Definition}
\newtheorem{cor}{Corollary}
\newtheorem{obs}{Observation}
\usepackage[compact]{titlesec}
\usepackage{dcolumn}
\usepackage{tikz}
\usetikzlibrary{arrows}
\usepackage{multirow}
\usepackage{xcolor}
\newcolumntype{.}{D{.}{.}{-1}}
\newcolumntype{d}[1]{D{.}{.}{#1}}
\definecolor{light-gray}{gray}{0.65}
\usepackage{url}
\usepackage{listings}
\usepackage{color}
\usepackage{rotating}
\usepackage{adjustbox}

\definecolor{codegreen}{rgb}{0,0.6,0}
\definecolor{codegray}{rgb}{0.5,0.5,0.5}
\definecolor{codepurple}{rgb}{0.58,0,0.82}
\definecolor{backcolour}{rgb}{0.95,0.95,0.92}

\lstdefinestyle{mystyle}{
	backgroundcolor=\color{backcolour},   
	commentstyle=\color{codegreen},
	keywordstyle=\color{magenta},
	numberstyle=\tiny\color{codegray},
	stringstyle=\color{codepurple},
	basicstyle=\footnotesize,
	breakatwhitespace=false,         
	breaklines=true,                 
	captionpos=b,                    
	keepspaces=true,                 
	numbers=left,                    
	numbersep=5pt,                  
	showspaces=false,                
	showstringspaces=false,
	showtabs=false,                  
	tabsize=2
}
\lstset{style=mystyle}
\newcommand{\Sref}[1]{Section~\ref{#1}}
\newtheorem{hyp}{Hypothesis}

\title{Problem Set 4}
\date{Due: December 3, 2023}
\author{Applied Stats/Quant Methods 1 \\ \vspace{\baselineskip}
	\textbf{Maiia Skrypnyk 23371609}}

\begin{document}
	\maketitle
	\section*{Instructions}
	\begin{itemize}
		\item Please show your work! You may lose points by simply writing in the answer. If the problem requires you to execute commands in \texttt{R}, please include the code you used to get your answers. Please also include the \texttt{.R} file that contains your code. If you are not sure if work needs to be shown for a particular problem, please ask.
		\item Your homework should be submitted electronically on GitHub.
		\item This problem set is due before 23:59 on Sunday December 3, 2023. No late assignments will be accepted.
	\end{itemize}



	\vspace{.5cm}
\section*{Question 1: Economics}
\vspace{.25cm}
\noindent 	
In this question, use the \texttt{prestige} dataset in the \texttt{car} library. First, run the following commands:

\begin{verbatim}
install.packages(car)
library(car)
data(Prestige)
help(Prestige)
\end{verbatim} 


\noindent We would like to study whether individuals with higher levels of income have more prestigious jobs. Moreover, we would like to study whether professionals have more prestigious jobs than blue and white collar workers.

\newpage
\begin{enumerate}
	
	\item [(a)]
	Create a new variable \texttt{professional} by recoding the variable \texttt{type} so that professionals are coded as $1$, and blue and white collar workers are coded as $0$ (Hint: \texttt{ifelse}).

	\lstinputlisting[language=R, firstline=40, lastline=42]{PS04MS.R} 

\begin{table}[H]
	\centering
	\caption{'Prestige' Dataset Header}
	\label{tab:prestige_dataset_header}
	\scriptsize
	\begin{adjustbox}{width=0.9\textwidth}
		\begin{tabular}{lccccccc}
			\begin{turn}{45}\textbf{Occupation}\end{turn} & \begin{turn}{45}\textbf{Education}\end{turn} & \begin{turn}{45}\textbf{Income}\end{turn} & \begin{turn}{45}\textbf{Women}\end{turn} & \begin{turn}{45}\textbf{Prestige}\end{turn} & \begin{turn}{45}\textbf{Census}\end{turn} & \begin{turn}{45}\textbf{Type}\end{turn} & \begin{turn}{45}\textbf{Professional}\end{turn} \\
			\hline
			Gov. Administrators & 13.11 & 12351 & 11.16 & 68.8 & 1113 & Prof & 1 \\
			General Managers & 12.26 & 25879 & 4.02 & 69.1 & 1130 & Prof & 1 \\
			Accountants & 12.77 & 9271 & 15.70 & 63.4 & 1171 & Prof & 1 \\
			Purchasing Officers & 11.42 & 8865 & 9.11 & 56.8 & 1175 & Prof & 1 \\
			Chemists & 14.62 & 8403 & 11.68 & 73.5 & 2111 & Prof & 1 \\
			Physicists & 15.64 & 11030 & 5.13 & 77.6 & 2113 & Prof & 1 \\
			Biologists & 15.09 & 8258 & 25.65 & 72.6 & 2133 & Prof & 1 \\
			Architects & 15.44 & 14163 & 2.69 & 78.1 & 2141 & Prof & 1 \\
			Civil Engineers & 14.52 & 11377 & 1.03 & 73.1 & 2143 & Prof & 1 \\
			Mining Engineers & 14.64 & 11023 & 0.94 & 68.8 & 2153 & Prof & 1 \\
			Surveyors & 12.39 & 5902 & 1.91 & 62.0 & 2161 & Prof & 1 \\
			Draughtsmen & 12.30 & 7059 & 7.83 & 60.0 & 2163 & Prof & 1 \\
			Computer Programmers & 13.83 & 8425 & 15.33 & 53.8 & 2183 & Prof & 1 \\
			Economists & 14.44 & 8049 & 57.31 & 62.2 & 2311 & Prof & 1 \\
			Psychologists & 14.36 & 7405 & 48.28 & 74.9 & 2315 & Prof & 1 \\
			Social Workers & 14.21 & 6336 & 54.77 & 55.1 & 2331 & Prof & 1 \\
			Lawyers & 15.77 & 19263 & 5.13 & 82.3 & 2343 & Prof & 1 \\
			Librarians & 14.15 & 6112 & 77.10 & 58.1 & 2351 & Prof & 1 \\
			Vocational Counsellors & 15.22 & 9593 & 34.89 & 58.3 & 2391 & Prof & 1 \\
			Ministers & 14.50 & 4686 & 4.14 & 72.8 & 2511 & Prof & 1 \\
			University Teachers & 15.97 & 12480 & 19.59 & 84.6 & 2711 & Prof & 1 \\
			Primary School Teachers & 13.62 & 5648 & 83.78 & 59.6 & 2731 & Prof & 1 \\
			Secondary School Teachers & 15.08 & 8034 & 46.80 & 66.1 & 2733 & Prof & 1 \\
			Physicians & 15.96 & 25308 & 10.56 & 87.2 & 3111 & Prof & 1 \\
			Veterinarians & 15.94 & 14558 & 4.32 & 66.7 & 3115 & Prof & 1 \\
			Osteopaths Chiropractors & 14.71 & 17498 & 6.91 & 68.4 & 3117 & Prof & 1 \\
			Nurses & 12.46 & 4614 & 96.12 & 64.7 & 3131 & Prof & 1 \\
			Nursing Aides & 9.45 & 3485 & 76.14 & 34.9 & 3135 & BC & 0 \\
			Physiotherapists & 13.62 & 5092 & 82.66 & 72.1 & 3137 & Prof & 1 \\
			Pharmacists & 15.21 & 10432 & 24.71 & 69.3 & 3151 & Prof & 1 
			\end{tabular}
		\end{adjustbox}
	\end{table}

	
	\item [(b)]
	Run a linear model with \texttt{prestige} as an outcome and \texttt{income}, \texttt{professional}, and the interaction of the two as predictors (Note: this is a continuous $\times$ dummy interaction.)
	
	\lstinputlisting[language=R, firstline=43, lastline=44]{PS04MS.R} 
	
Exploring the model's summary and using 'stargazer' function to format results: 
	\lstinputlisting[language=R, firstline=45, lastline=46]{PS04MS.R} 
	
	\begin{table}[H] \centering  \caption{Linear Model 1}
		\label{tab:model1}\begin{tabular}{@{\extracolsep{5pt}}lc} \\[-1.8ex]\hline \hline \\[-1.8ex]  & \multicolumn{1}{c}{\textit{Dependent variable:}} \\ \cline{2-2} \\[-1.8ex] & prestige \\ \hline \\[-1.8ex]  income & 0.003$^{***}$ \\   & (0.0005) \\   & \\  professional & 37.781$^{***}$ \\   & (4.248) \\   & \\  income:professional & $-$0.002$^{***}$ \\   & (0.001) \\   & \\  Constant & 21.142$^{***}$ \\   & (2.804) \\   & \\ \hline \\[-1.8ex] Observations & 98 \\ R$^{2}$ & 0.787 \\ Adjusted R$^{2}$ & 0.780 \\ Residual Std. Error & 8.012 (df = 94) \\ F Statistic & 115.878$^{***}$ (df = 3; 94) \\ \hline \hline \\[-1.8ex] \textit{Note:}  & \multicolumn{1}{r}{$^{*}$p$<$0.1; $^{**}$p$<$0.05; $^{***}$p$<$0.01} \\ \end{tabular} \end{table} 
		
	\item [(c)]
	Write the prediction equation based on the result.
	
A general formula for such equation would be: 
\[ Y_i = \beta_0 + \beta_1 X_i + \beta_2 D_i + \beta_3 X_i D_i + \varepsilon_i \]


\begin{flushleft}
	\begin{itemize}
		\item $\textit{Y}_{i}$ = predicted value of the response variable 
		\item $\beta_0$ = estimated intercept (the expected value of Y when X=0 and D=0)
		\item $\beta_1 $ = estimated slope coefficient (the change in Y when X increases by one unit, when D=0)
		\item $\beta_2$ = estimated slope coefficient (the change in Y when D increases by one unit, when X=0)
		\item $\beta_3$ = estimated slope coefficient (the interaction term of X and D)
		\item $\textit{X}_{i}$ = (any chosen) value of the explanatory variable 
		\item $\textit{D}_{i}$ = (any chosen) value of the moderator variable (= dummy variable, in our case)
		
	\end{itemize}
\end{flushleft}

Therefore, a prediction equation for Model 1 will look like this:

\[ \hat{\textbf{\textit{Prestige}}}_i = \hat{\beta}_0 + \hat{\beta}_1 \textit{Income}_i + \hat{\beta}_2 \textit{Professional}_i + \hat{\beta}_3 \textit{Income}_i \textit{Professional}_i + \varepsilon_i \]

\[ \hat{\textbf{\textit{Prestige}}}_i = 21.142 + (0.003 \cdot \textit{Income}_i) + (37.781 \cdot \textit{Professional}_i) - (0.002 \cdot \textit{Income}_i \cdot \textit{Professional}_i)+ \varepsilon_i \]


	\item [(d)]
	Interpret the coefficient for \texttt{income}.
	
1) Holding 'professionalism' (= the '\textit{Professional}' variable) constant, a one unit (= one US dollar) increase in income is, on average, associated with a 0.003 scale point increase in prestige. 

2) For non-professionals (blue and white collar workers; \textit{Professional }= 0), a one unit (= one US dollar) increase in income is, on average, associated with a 0.003 scale point increase in prestige. 

This estimated coefficient is statistically differentiable from 0 (= significant) at the \(\alpha\) level = 0.01.

	\vspace{0.5cm}	
	\item [(e)]
	Interpret the coefficient for \texttt{professional}.

1) Holding income constant, being a professional (\textit{Professional} = 1) is associated with having 37.781 scale points of prestige higher as compared to a non-professional (\textit{Professional} = 0).

2) Holding income constant, a change from being a non-professional to becoming a professional (\textit{Professional} goes from 0 to 1) is associated with, on average, 37.781 scale points increase of prestige.

3) For professionals with zero income, prestige is estimated as being, on average, by 37.781 scale points higher in comparison with non-professionals with zero income.

This estimated coefficient is statistically differentiable from 0 (= significant) at the \(\alpha\) level = 0.01.

	\item [(f)]
	What is the effect of a \$1,000 increase in income on prestige score for professional occupations? In other words, we are interested in the marginal effect of income when the variable \texttt{professional} takes the value of $1$. Calculate the change in $\hat{y}$ associated with a \$1,000 increase in income based on your answer for (c).
	
1) Given \(Y = \text{Prestige}\), \(D = \text{Professional} = 1\), \(X = \text{Income} = 0\):

\[ \hat{\textit{y}}_0 = \hat{\beta}_0 + (\hat{\beta}_1 \cdot 0) + (\hat{\beta}_2 \cdot 1) + (\hat{\beta}_3 \cdot 0 \cdot 1)  \] 

\[ \hat{\textit{y}}_0 = 21.142  + (0.003 \cdot 0) + (37.781 \cdot 1) + (-0.002 \cdot 0 \cdot 1)  \] 

\[ \hat{\textit{y}}_0 = 21.142 + 37.781 = 58.923 \] 

2) Given \(Y = \text{Prestige}\), \(D = \text{Professional} = 1\), \(X = \text{Income} = 1000\):

\[ \hat{\textit{y}}_1 = \hat{\beta}_0 + (\hat{\beta}_1 \cdot 0) + (\hat{\beta}_2 \cdot 1) + (\hat{\beta}_3 \cdot 0 \cdot 1)  \] 

\[ \hat{\textit{y}}_1 = 21.142  + (0.003 \cdot 1000) + (37.781 \cdot 1) + (-0.002 \cdot 1000 \cdot 1)  \] 

\[ \hat{\textit{y}}_1 = 21.142 + 3 + 37.781 - 2 = 59.923 \] 


3) Therefore, we can calculate the change in $\hat{\textit{y}}$ as the difference between the two:

\textbf{$\hat{y}_1 - \hat{y}_0$ }= 59.923 - 58.823 = 1

4) The other way to calculate this value would be:

Change in \textbf{$\hat{y}$} = $(\hat{\beta}_1 + \hat{\beta}_3) \cdot \Delta X$ (where $\Delta X$, or change in income, is \$1,000)

Change in \textbf{$\hat{y}$} = $(0.003 - 0.002) \cdot 1000 = 1$

So, for professional occupations, on average, the change in $\hat{\textit{y}}$ associated with a \$1,000 increase in income equals 1 scale point of prestige.
	
	\item [(g)]
	What is the effect of changing one's occupations from non-professional to professional when her income is \$6,000? We are interested in the marginal effect of professional jobs when the variable \texttt{income} takes the value of $6,000$. Calculate the change in $\hat{y}$ based on your answer for (c).


1) Given \(Y = \text{Prestige}\), \textit{D} = \text{Professional}\ = 0, \(X = \text{Income} = 6000\):

\[ \hat{\textit{y}}_0 = \hat{\beta}_0 + (\hat{\beta}_1 \cdot 6000) + (\hat{\beta}_2 \cdot 0) + (\hat{\beta}_3 \cdot 6000 \cdot 0)  \] 

\[ \hat{\textit{y}}_0 = 21.142  + (0.003 \cdot 6000) + (37.781 \cdot 0) + (-0.002 \cdot 6000 \cdot 0)  \] 

\[ \hat{\textit{y}}_0 = 21.142 + 18 = 39.142 \] 

2) Given \(Y = \text{Prestige}\), \(D = \text{Professional} = 1\), \(X = \text{Income} = 6000\):

\[ \hat{\textit{y}}_1 = \hat{\beta}_0 + (\hat{\beta}_1 \cdot 0) + (\hat{\beta}_2 \cdot 1) + (\hat{\beta}_3 \cdot 0 \cdot 1)  \] 

\[ \hat{\textit{y}}_1 = 21.142  + (0.003 \cdot 6000) + (37.781 \cdot 1) + (-0.002 \cdot 6000 \cdot 1)  \] 

\[ \hat{\textit{y}}_1 = 21.142 + 18 + 37.781 - 12 = 64.923 \] 

3) Therefore, we can calculate the change in $\hat{\textit{y}}$ as the difference between the two:

\textbf{$\hat{y}_1 - \hat{y}_0$ }= 64.923 - 39.142 = 25.781

4) The other way to calculate this value would be:

Change in \textbf{$\hat{y}$} = $\hat{\beta}_2 + \hat{\beta}_3 \cdot \textit{Income}_i$ (where $\textit{Income}_i$ is \$6,000)


Change in \textbf{$\hat{y}$} = $37.781 - 0.002 \cdot 6000 = 37.781 - 12 = 25.781 $

So, holding income constant at \$6,000, a change from being a non-professional to becoming a professional (\textit{Professional} goes from 0 to 1) is associated with, on average, 25.781 scale points increase of prestige.

\end{enumerate}

\section*{Question 2: Political Science}
\vspace{.25cm}
\noindent 	Researchers are interested in learning the effect of all of those yard signs on voting preferences.\footnote{Donald P. Green, Jonathan	S. Krasno, Alexander Coppock, Benjamin D. Farrer,	Brandon Lenoir, Joshua N. Zingher. 2016. ``The effects of lawn signs on vote outcomes: Results from four randomized field experiments.'' Electoral Studies 41: 143-150. } Working with a campaign in Fairfax County, Virginia, 131 precincts were randomly divided into a treatment and control group. In 30 precincts, signs were posted around the precinct that read, ``For Sale: Terry McAuliffe. Don't Sellout Virgina on November 5.'' \\

Below is the result of a regression with two variables and a constant.  The dependent variable is the proportion of the vote that went to McAuliff's opponent Ken Cuccinelli. The first variable indicates whether a precinct was randomly assigned to have the sign against McAuliffe posted. The second variable indicates
a precinct that was adjacent to a precinct in the treatment group (since people in those precincts might be exposed to the signs).  \\

\vspace{.5cm}
\begin{table}[!htbp]
	\centering 
	\textbf{Impact of lawn signs on vote share}\\
	\begin{tabular}{@{\extracolsep{5pt}}lccc} 
		\\[-1.8ex] 
		\hline \\[-1.8ex]
		Precinct assigned lawn signs  (n=30)  & 0.042\\
		& (0.016) \\
		Precinct adjacent to lawn signs (n=76) & 0.042 \\
		&  (0.013) \\
		Constant  & 0.302\\
		& (0.011)
		\\
		\hline \\
	\end{tabular}\\
	\footnotesize{\textit{Notes:} $R^2$=0.094, N=131}
\end{table}

\newpage 

Before we start analysing the model, let's annotate the coefficients:

\begin{flushleft}
	\begin{itemize}
		\item $\hat{\beta}_0$ (constant/intercept) = 0.302 (SE = 0.011)
		\item $\hat{\beta}_1$ (estimated slope coefficient) = 0.042 (SE = 0.016)
		\item $\hat{\beta}_2$ (estimated slope coefficient) = 0.042 (SE = 0.013)
	\end{itemize}
\end{flushleft}

\vspace{.5cm}
\begin{enumerate}
	\item [(a)] Use the results from a linear regression to determine whether having these yard signs in a precinct affects vote share (e.g., conduct a hypothesis test with $\alpha = .05$).
	
	As we need to determine the partial effect of individual regression coefficients, I am using a Student's t-test to conduct a hypothesis test in both (a) and (b).
	
1) \textbf{Assumptions: } categorical data, randomly obtained, normally distributed.

2) \textbf{Null and Alternative Hypotheses:}
\begin{itemize}
	\item \(H_0: \hat{\beta}_1 = 0\) (There is NO association between the proportion of vote share that went to Cuccinelli and whether a precinct was randomly assigned to display signs against McAuliffe);
	\item \(H_a: \hat{\beta}_1 \neq 0\) (There IS an association...).
\end{itemize}

3)\textbf{ T-statistic }is given by:

\[ t = \frac{\hat{\beta}_1 - \hat{\beta}_0}{SE}} \]

\[ t = \frac{0.042 - 0}{0.016} = 2.625 \]

4) \textbf{P-value: } 
\begin{itemize}
 \item two-tailed probability from t-distribution; 
 \item DF = \textit{n - k - p }= 131 - 2 - 1 = 128;
 \item \( p = 2 \times Pr(t_{128} > |2.625|) \) = 0.0097
 
 \lstinputlisting[language=R, firstline=51, lastline=59]{PS04MS.R} 
 
\newpage 
\textbf{\textit{Interpretation:}} Since the p-value of 0.0097 is less than the significance level of 0.05, we have found statistically significant evidence to \textbf{reject }the null hypothesis that there is no association between the proportion of vote share that went to Cuccinelli and whether a precinct was randomly assigned to display signs against McAuliffe (after controlling for the effects of the other predictor variable in the model).

	\item [(b)]  Use the results to determine whether being
	next to precincts with these yard signs affects vote
	share (e.g., conduct a hypothesis test with $\alpha = .05$).

1) \textbf{Assumptions: }categorical data, randomly obtained, normally distributed.

2) \textbf{Null and Alternative Hypotheses:}

\begin{itemize}
	\item[$\bullet$] \(H_0: \hat{\beta}_2 = 0\) (There is NO association between the proportion of vote share that went to Cuccinelli and whether a precinct was adjacent to the precincts that were randomly assigned to display signs against McAuliffe);
	\item[$\bullet$] \(H_a: \hat{\beta}_2 \neq 0\) (There IS an association...).
\end{itemize}


3)\textbf{ T-statistic }is given by:

\[ t = \frac{\hat{\beta}_2 - \hat{\beta}_0}{SE}} \]

\[ t = \frac{0.042 - 0}{0.013} \approx 3.23  \]

4) \textbf{P-value: } 

\begin{itemize}
\item [$\bullet$]  two-tailed probability from t-distribution; 
\item [$\bullet$]  DF = \textit{n - k - p }= 131 - 2 - 1 = 128;
\item [$\bullet$]  \( p = 2 \times Pr(t_{128} > |3.23|) \) = 0.00157 $\approx$ 0.0016

\end{itemize}

\lstinputlisting[language=R, firstline=62, lastline=68]{PS04MS.R} 

\newpage 
\textbf{\textit{Interpretation:}} Since the p-value of $\approx$ 0.0016 is less than the significance level of 0.05, we have found statistically significant evidence to \textbf{reject }the null hypothesis that there is no association between the proportion of vote share that went to Cuccinelli and whether a precinct was adjacent to the precincts that were randomly assigned to display signs against McAuliffe (after controlling for the effects of the other predictor variable in the model)..

	\item [(c)] Interpret the coefficient for the constant term substantively.
	
	The coeffiecient for the constant term (intercept) represents the expected proportion of vote share (0.302) that went to Cuccinelli in precincts that neither had yard signs nor were adjacent to the ones that had yard signs (control group with no influence of external factors); 'baseline' vote share.
	
	\item [(d)] Evaluate the model fit for this regression.  What does this	tell us about the importance of yard signs versus other factors that are not modeled?
	
	To evaluate the model fit, we should pay attention to the correlation coefficient \( R^2 = 0.094 \). It's value is positive though quite low (compared to max = 1), which suggests that only 9.4\%
	of variability in vote share for Cuccinelli is explained by precincts having or being by implication exposed to the yard signs. While the effect of signs on vote share is statistically significant, there are still a lot of other factors that should have been accounted for in the model to be fully comprehensive. Such factors might be gender, race, age, religiosity, income, occupation, etc., etc...
	
\end{enumerate}  


\end{document}
